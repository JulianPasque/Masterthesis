\chapter{Cross Plattform Frameworks}
Für die Realisierung eine Source-to-Source Compilers gibt es folglich zwei Faktoren,  die für die Realisierung ausschlaggebend sind.  Zum einen die Programmiersprachen in dem die beiden Frameworks entwickelt werden,  da bei der Übersetzung eine Brücke für die Übersetzung zwischen Quell und Zielsprache verwendet werden muss.  Neben der Programmiersprache ist jedoch auch die Arbeitsweise eines Frameworks von essentieller Bedeutung.  Wie die Definition von Compilern bereits einführte,  müssen die Programme vor und nach der Übersetzung gleichwertig sein.  Dies implementiert,  dass das Verhalten der übersetzten Anwendungen nach der Übersetzung identisch seien muss wie das der Ursprungsanwendung.  Es ist also notwendig,  neben den sprachlichen auch die technischen Unterschiede zwischen den Frameworks zu kennen und diese im Rahmen der compilierung zu optimieren. 
\section{Programmiersprachen}
C\# Dart Vergleich
XAML Dart
\section{Frameworks}
Xamarin Forms und Flutter vergleichen
\subsection{Projekte}
Files.
Xamarin.Forms Anwendungen setzen sich aus mehreren Projekten zusammen.  Dabei gibt es klassischerweise für jede Zielpalttform ein Projekt sowie ein zusätzliches für den Quelltext, der zwischen den Plattformen geteilt wird.  Dabei ist das geteilte Projekt eine Abhängigkeit der plattformspezifischen Projekte.  Im Gegensatz dazu gibt es bei Flutter nur ein Projekt, welches alle notwendigen Inhalte für iOS und Android beinhaltet. 
\subsection{Ansichtsseiten}
Ansichtseiten sind visuelle elemente einer Anwendung die den gesamten Bildschirm belegen. (https://docs.microsoft.com/de-de/xamarin/xamarin-forms/user-interface/controls/pages) Bei iOS werden diese klaissischerweise als Viewcontroller und Acitvity bei Android bezeichnet.   Xamarin Forms bietet dafür ein Vielzahl von alternativen an: Eine Contentpage, eine MasterDetailPage, eine NavigationPage,  eine TabbedPage eine TempaltedPage und eine CarouselPage.
Die Auswahl einer Ansischtsseite hat daher einen direkten Einfluss aus die Navigation des Anwenders durch die Verschiedenen Seiten.
Bei Flutter gibt es alternativ weniger Auswahl, da lediglich ein "MaterialApp" und eine "CupertinoApp" als Widget zur Verfügung stellen und alternativ Materialdesign oder Cupertino design anbieten. 
Um Seiten mit Tab oder einem Slideout zu genierieren bietet Flutter jedoch ebenfalls Widgets an. 
\subsection{Elemente}
\subsection{Layouts}
\subsection{Listen}
\subsection{Gesten}
\subsection{Animtion}

\subsection{Navigation}
Pages.
Andere Apps
\subsection{Async UI}
\subsection{Netzwerk Anfragen}
\subsection{Abhängigkeiten}
\subsection{Lebenzyklus}
\subsection{Schriften}
\subsection{Plugins}
Interacting with hardware, third party services, and the platform
\subsection{Databases and locale storage}
\subsection{Notifications}


