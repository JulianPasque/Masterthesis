\chapter{Unterschiede zwischen C\# und Dart}
\label{chap:Programmiersprachen}

Neben den in Kapitel \ref{chap:CrossPlattformFrameworks} behandelten unterschieden zwischen den Frameworks werden in diesem Kapitel die Unterschiede zwischen den Hochsprachen \Csharp und Dart behandelt.  Die beiden Programmiersprachen haben einen ähnlichen Stils und eine vergleichbare Syntax was den Umstieg für Programmierer zu Dart vereinfachen sollte.  \footcite[Vgl. ][Abgerufen am \today]{Pedley2019} Potentiell haben alle der hier behandelten Aspekte einen Einfluss auf die mit den Frameworks mobilen Anwendungen da diese Sprachelemente in dem Quelltext der Projekte vorkommen kann.  

\section{Null Sicherheit}
In \Csharp unterteilt man zwischen Wertetypen und Referenztypen.  Eine Variable eines Wertetyps enthält eine Instanz des Typs.  Dies unterscheidet sich von einer Variablen eines Referenztyps, die eine Referenz auf eine Instanz des Typs enthält.  Standardmäßig werden bei einer Zuweisung,  der Übergabe eines Arguments an eine Methode und der Rückgabe eines Methodenergebnisses die Variablenwerte kopiert. Bei Variablen von Wertetypen werden die entsprechenden Typinstanzen kopiert.  \Csharp bietet die folgenden eingebauten Wertetypen: Ganzzahlige numerische Typen (Integer) Fließkomma numerische Typen (Float und Double), boolean und Char welcher ein Zeichen darstellt.  Jeder Variablen dieser  Typen muss immer einen zum Typ passender Wert zugewiesen sein.  So muss zu jeder Zeit in einer Instanz des Types Integer eine Zahl verwaltet. \footcite[Vgl. ][Abgerufen am \today]{MicrosoftValueTypes2020} Um die Notwendigkeit eines Wertes zu umgehen gibt es in der Hochsprache die sogenannte nullable Typen die neben den zulässigen Werten zusätzlich den Wert null zulassen. \footcite[Vgl.][S. 167]{Bayer2008} 

Die Programmiersprache Dart war klassischerweise nicht "Null sicher".  Bis zum März 2021 waren alle Variablen in Dart potentiell null,  das hatte zur Konsequenz, dass jede Variable potentiell keinen Wert hinterlegt hatte.   Das galt auch für Vergleichbare Typen der aus \Csharp bekannten Wertetyp.  Dies führte dazu,  dass bei der Arbeit Variablen Null- Prüfung ,dargestellt in Quelltext \ref{lst:DartNull},  durchgeführt werden musste um Fehler während der Laufzeit zu vermeiden.

\lstinputlisting[label={lst:DartNull},caption={[Null-Sicherheit in Dart 1.x]{Null-Sicherheit in Dart 1.x\footcite[In Anlehnung an ][Abgerufen am \today]{Pedley2019}}}, language=Dart]{SourceCode/DartNull.Dart}

Im März 2021 hat\footcitetext[Quelltext in Anlehnung an][Abgerufen am \today]{Pedley2019} die Programmiersprache Dart den Support für Null-Sicherheit hinzugefügt.  So können Typen, wie in \Csharp nicht den Wert null erhalten,  es sei den der Entwickler entscheidet sich bewusst für das Gegenteil.  Mit Null Sicherheit werden die Laufzeit-Nullreferenzfehler zu Analysefehlern zur Bearbeitungszeit die nicht mehr Zwangsläufig zum Absturz der Anwendung führen. \footcite[Vgl.][Abgerufen am \today]{GoogleflutterNullSafty2021} Quelltext \ref{lst:DartNotNullAnymore} visualisiert die Arbeit mit den neuen Datentypen in Dart. 

\lstinputlisting[label={lst:DartNotNullAnymore},caption={[Null-Sicherheit in Dart 2.x]{Null-Sicherheit in Dart 2.x\footcite[In Anlehnung an ][Abgerufen am \today]{GoogleflutterNullSafty2021}}}, language=Dart]{SourceCode/DartNullable.Dart}

Nach\footcitetext[Quelltext in Anlehnung an][Abgerufen am \today]{GoogleflutterNullSafty2021} dieser Veränderung verhält sich die Programmiersprache Dart identisch wie \Csharp  sogar das Fragezeichen für die Definition einer Variable die Potentiell null seien kann ist identisch.  Durch diese Veränderung muss der Compiler keine speziellen Änderungen mehr vornehmen,  der Vollständigkeit halber sollte dieser Änderung trotzdem nicht unerwähnt bleiben.

\section{Datentypen}

Das Verständnis des Unterschieds zwischen Referenz und Wertetypen ist elementar für die Programmierung mit .NET.  Ein Unterschied besteht in der Initialisierung von Referenztypen.  Während es bei Werttypen einfach so ist, dass der entsprechende Wert zugewiesen werden kann, muss bei Referneztyen expliziert ein Objekt erzeugt oder ein bestehendes Objekt zugewiesen werden.  Die Erzeugung eines neues Objekts geschieht dabei mit dem Operator New. \footcite[Vgl.][S. 93]{Kofler2005} In Dart muss dieser Operator nicht verwendet werden, da Dart herausfindet, wann ein neues Objekt initiiert werden muss.\footcite[Vgl. ][Abgerufen am \today]{GoogleFlutterTour2020} Dies wird in Quelltext \ref{lst:DartNew} dargestellt. 

\lstinputlisting[label={lst:DartNew},caption={[Optionales "New" Keyword in Dart]{Optionales "New" Keyword in Dart\footcite[Quelltext in Anlehnung an][Abgerufen am \today]{Pedley2019}}}, language=Dart]{SourceCode/DartNew.Dart}

Nach diesem wesentlichen\footcitetext[Quelltext in Anlehnung an][Abgerufen am \today]{Pedley2019} Unterschied kann nun ein Vergleich (siehe Tabelle \ref{tab:Datatypes}) zwischen den verfügbaren Datentypen von \Csharp und Dart betrachtet werden. 
\begin{table}[!ht]
\begin{tabularx}{\textwidth}{X|X|X}
   \textbf{Datentyp} & \textbf{\Csharp}  &  \textbf{Dart}  \\
\hline
	Ganzzahl            			&  sbyte    		& Integer \\ 
										&	byte 			& 				\\ 
										&	short 			& 				\\ 
										&	ushort 			& 				\\ 
										&	int 			& 				\\ 
										&	uint 			& 				\\ 
										&	long 			& 				\\ 
										&	ulong 			& 				\\ 
	\hline
	Fließkommazahlen         &  double 			& Double \\
										&	float 			& 				\\ 
										&	decimal 			& 				\\ 
	\hline
	Strings      					&  String        	& String 					\\ 
	\hline
	Textzeichen      			&  char        	&  					\\ 
	\hline
	Array      						&  Array        	&  					\\ 
	\hline
	Booleans            			& 	bool				& bool \\ 
	\hline
	Lists          					& List				& List \\ 
	\hline
	Maps            					& Dictionairy		& Map \\ 
\end{tabularx}
\caption{Gegenüberstellung Datentypen}
 \label{tab:Datatypes}
\end{table}
Bis auf den Datentyp String handelt es sich bei der Anzeige ausschließlich um Wertetypen, String ist ausschließlich auf Grund seiner häufigen Verwendung teil dieses Vergleiches.  Wie die Tabelle zeigt, stehen in Flutter im Gegensatz zu \Csharp  keine Auswahl an verschiedenen Ganzzahl Typen, von 8 bis 64 Bit,  mit und ohne Vorzeichen zur Verfügung.  In der Praxis gibt es aber auch Dart unterschiedliche interne Darstellungen,  je nachdem,  welcher Integer-Wert zur Laufzeit tatsächlich verwendet wird.  Ein weiterer Unterschied ist die nicht Verfügbarkeit von Arrays in Dart.  Stattdessen wird auf Listen zurückgegriffen, wann immer ein Array verwendet werden würde. \footcite[Vgl. ][Abgerufen am \today]{Ford2019}

In Flutter Die standardmäßige erweiterbare Liste, wie sie von [] erstellt wird, behält einen internen Puffer und erweitert diesen Puffer bei Bedarf. Dies garantiert, dass eine Folge von Additionsoperationen jeweils in amortisierter konstanter Zeit ausgeführt wird. Das direkte Setzen der Länge kann Zeit in Anspruch nehmen, die proportional zur neuen Länge ist, und kann die interne Kapazität verändern, so dass eine nachfolgende Add-Operation die Pufferkapazität sofort erhöhen muss

\Csharp und .Net haben Klassen die für die Behandlung von Sammlungen (engl. Collection).   
Listen sind Auflistungen mit einem Integer Index Zugriff.  Diese Auflistungen können verwendet werden, wenn dynamische Auflistungen verwaltet werden sollen. 
 Diesedie alle Auftretenden Probleme des Hinzufügen und entfernen von Elementen aus Arrays behandeln.  Diese werden als Listen (Eng. List) bezeichnet. \footcite[Vgl.][S. 413]{Stellman2021} Dart hat ebenfalls Listen die unter dem gleichen Namen zur Verfügung stehen und die gleichen Funktionalitäten abbilden. \footcite[Vgl.][S. 12f ]{Meiller2020}
Listen sind Listen, und Dictionaries sind Maps.

\Csharp und .Net haben Sammlungen (Collection) Klassen die alle Auftretenden Probleme des Hinzufügen und entfernen von Elementen aus Arrays behandeln.  Diese werden als Listen (Eng. List) bezeichnet. \footcite[Vgl.][S. 413]{Stellman2021} Dart hat ebenfalls Listen die unter dem gleichen Namen zur Verfügung stehen und die gleichen Funktionalitäten abbilden. \footcite[Vgl.][S. 12f ]{Meiller2020}
Listen sind Listen, und Dictionaries sind Maps.


\lstinputlisting[label={lst:DartDicAndList}, caption={[Listen und Maps in Dart]{Listen und Maps in Dart\footcite[Quelltext in Anlehnung an][Abgerufen am \today]{Pedley2019}}},  language=Dart]{SourceCode/DartListAndDictionairy.Dart}

\section{Modifizierer}

Alle \footcitetext[Quelltext in Anlehnung an][Abgerufen am \today]{Pedley2019} Typen und Typmember verfügen in beiden Sprachen über eine Zugriffsebene.  Diese Zugriffsebene steuert, ob sie von anderem Code in Ihrer Assembly oder anderen Assemblys verwendet werden können.  Dabei wird in \Csharp mit Mithilfe der Zugriffsmodifizieren (public,  private,  protected,  internal,  protected internal,  private protected) der Zugriff auf einen Typ oder Member festlegen.  In Dart gibt es die Schlüsselwörter public,  protected und private nicht. Wenn ein Bezeichner mit einem Unterstrich (\_) beginnt, ist er für seine Bibliothek privat dies wird in \ref{lst:PrivatePublicDart} als Quelltext dargestellt. 

\lstinputlisting[label={lst:PrivatePublicDart},caption={[Private und Public Definitionen in Dart]{Private und Public Definitionen in Dart\footcite[Quelltext in Anlehnung an][Abgerufen am \today]{Pedley2019}}}, language=Dart]{SourceCode/PrivateDefinition.Dart}

Da ein \footcitetext[Quelltext in Anlehnung an][Abgerufen am \today]{Pedley2019} Unterstrich ein valides Zeichen bei der Typ und Typmemberdefinition ist, ist daher bei der Übersetzung darauf zu achten,  dass existierende Unterstriche entfernt werden müssen da dies ansonsten potentiell zu fehlerhaften Übersetzungen führen kann.  Außerdem kann es durch die Verwendung von Zugriffsmodifiziern wie "protected internal" vorkommen, dass es keine entsprechende Dart Implementierung existiert.  Folglich führt dies potentiell zu falschen Zugriffsbeschränkungen bei der Übersetzung von Bibliotheken. 

\section{Vererbung}

Die Vererbung ist, zusammen mit der Kapselung und der Polymorphie, eines der drei primären Charakteristika des objektorientierten Programmierens. Die Vererbung ermöglicht die Erstellung neuer Klassen, die in anderen Klassen definiertes Verhalten wieder verwenden, erweitern und ändern. Die Klasse, deren Member vererbt werden, wird Basisklasse genannt, und die Klasse, die diese Member erbt, wird abgeleitete Klasse genannt. Eine abgeleitete Klasse kann nur eine direkte Basisklasse haben.\footcite[Vgl.  ][Abgerufen am \today]{GoogleFlutterSharedPreferences2020} 

Eine Schnittstelle definiert einen Vertrag. Jede class oder struct, die diesen Vertrag implementiert, muss eine Implementierung der in der Schnittstelle definierten Member bereitstellen. Ab \Csharp 8.0 kann eine Schnittstelle eine Standardimplementierung für Member definieren.\footcite[Vgl. interface (\Csharp-Referenz)][Abgerufen am \today]{GoogleFlutterSharedPreferences2020} 

Dart hat keine Schnittstellen, Sie haben abstrakte Klassen. Sie implementieren abstrakte Klassen. Wenn Sie erben wollen, erweitern Sie Klassen.

\lstinputlisting[label={lst:DartInherit},caption={[Vererbung in Dart]{Vererbung in Dart\footcite[Quelltext in Anlehnung an][Abgerufen am \today]{Pedley2019}}},  language=Dart]{SourceCode/DartInherit.Dart}

Sie \footcitetext[Quelltext in Anlehnung an][Abgerufen am \today]{Pedley2019} können eine Klasse erweitern und mehrere Klassen implementieren. Dart unterstützt auch Mixins.  Ein Mixin ist wie das Anhängen einer Klasse und das Hinzufügen ihrer Funktionalität zu der Klasse, ohne tatsächlich von ihr zu erben. Dies ist auch ähnlich wie die Schnittstellenimplementierungen von \Csharp 8.0.

\lstinputlisting[label={lst:DartMixin},caption={[Mixin's in Dart]{Mixin's in Dart\footcite[Quelltext in Anlehnung an][Abgerufen am \today]{Pedley2019}}}, language=Dart]{SourceCode/DartMixin.Dart}

\section{Namespaces}

Namespaces\footcitetext[Quelltext in Anlehnung an][Abgerufen am \today]{Pedley2019} werden häufig in \Csharp -Programmen auf zwei verschiedene Arten verwendet. Erstens: Die .NET-Klassen verwenden Namespaces, um ihre zahlreichen Klassen zu organisieren. Zweitens: Eigene Namespaces zu deklarieren kann Ihnen dabei helfen, den Umfang der Klassen- und Methodennamen in größeren Programmierprojekten zu steuern.
Die meisten \Csharp -Anwendungen beginnen mit einem Abschnitt von using-Anweisungen. Dieser Abschnitt enthält die von der Anwendung häufig verwendeten Namespaces und erspart dem Programmierer die Angabe eines vollqualifizierten Namens bei jedem Verwenden einer enthaltenen Methode.\footcite[Vgl.  Verwenden von Namespaces (\Csharp-Programmierhandbuch)][Abgerufen am \today]{GoogleFlutterSharedPreferences2020} 

Dart hat keine Namespaces. Stattdessen importieren Sie Pakete oder Dateien als solche.
Dadurch haben Sie Zugriff auf alle Klassen und Funktionen innerhalb der Datei. Aber wenn es einen Namenskonflikt gibt oder Sie die Dinge etwas lesbarer machen wollen, können Sie sie benennen.

\lstinputlisting[label={lst:DartPackages},caption={[Importieren von Paketen in Dart]{Importieren von Paketen in Dart\footcite[[In Anlehnung an ][Abgerufen am \today]{Pedley2019}}}, language=Dart]{SourceCode/DartPackages.Dart}

\footcitetext[Quelltext in Anlehnung an][Abgerufen am \today]{Pedley2019}


\section{Generics}
Generics  wurden in .NET als Typparameter eingeführt,  wodurch Klassen und Methoden entworfen werden können, bei denen ein Typ erst verzögert werden kann,  bis die Klasse oder Methode vom Clientcode deklarierst und instanziiert wird.  So kann indem z. B. ein generischen Typparameter „T“ verwendet wird,  eine einzelne Klasse geschrieben werden, die von unterschiedlichen Methoden verwendet wird, ohne dass Kosten und Risiken durch die Umwandlungen zur Laufzeit anfallen.\footcite[Vgl. ][Abgerufen am \today]{MicrosoftGenerics2015} 

Generics werden in Dart sehr ähnlich behandelt wie in \Csharp ,  mit der Ausnahme,  dass Sie eine generische Klasse ohne die Type-Beschränkung übergeben können.  \footcite[Vgl.][S. 98]{Cheng2019} Quelltext \ref{lst:DartGeneric} zeigt die Implementation einer generischen State-Klasse in Dart. 

\lstinputlisting[label={lst:DartGeneric}, caption={[Generics in Dart]{Generics in Dart\footcite[Quelltext in Anlehnung an][Abgerufen am \today]{Pedley2019}}}, language=Dart]{SourceCode/DartGeneric.Dart}
\section{Delegates}

In .Net\footcitetext[Quelltext in Anlehnung an][Abgerufen am \today]{Pedley2019} ist Ein Delegate ein Typ,  der Verweise auf Methoden mit einer bestimmten Parameterliste und dem Rückgabetyp darstellt.  Nach der Instanziierung eines Delegaten können die Instanz mit einer beliebigen Methode verknüpft werden,  die eine kompatible Signatur und einen kompatiblen Rückgabetyp aufweisen.  Diese können die Methode über die Delegatinstanz aufrufen.  Delegates werden dazu verwendet,  um Methoden als Argumente an anderen Methoden zu übergeben.  Da Ereignishandler ebenfalls Methoden sind können diese durch Delegates aufgerufen werden.  Benutzerdefinierte Methoden können also durch Steuerelemente diese Methode aufrufen, wenn ein bestimmes Ereignis wie ein klick auf einen Button eintritt.\footcite[Vgl. ][Abgerufen am \today]{MicrosoftDelegates2015} 

In Dart kann der Typ Typedef verwendet werden, um eine Methodensignatur zu definieren und eine Instanz davon in einer Variablen zu halten. \footcite[Vgl. ][Abgerufen am \today]{Pedley2019}  Dies wird in Quelltext \ref{lst:DartDelegates} dargestellt. 


\lstinputlisting[label={lst:DartDelegates},  caption={[Delegates in Dart]{Delegates in Dart\footcite[Quelltext in Anlehnung an][Abgerufen am \today]{Pedley2019}}}, language=Dart]{SourceCode/DartDelegates.Dart}


\section{Bibliotheken}

Klassenbibliotheken\footcitetext[Quelltext in Anlehnung an][Abgerufen am \today]{Pedley2019} sind das Konzept der freigegebenen Bibliothek für .NET. Sie können damit nützliche Funktionalität auf Module verteilen, die von mehreren Anwendungen verwendet werden können. Sie können auch verwendet werden, um Funktionalität zu laden, die beim Start der Anwendung nicht benötigt wird bzw. nicht bekannt ist. Klassenbibliotheken werden mithilfe des .NET Assembly-Dateiformats beschriebenen.\footcite[Vgl.  .NET-Klassenbibliotheken][Abgerufen am \today]{GoogleFlutterSharedPreferences2020} 

Dart verfügt über eine Vielzahl von Kernbibliotheken, die für viele alltägliche Programmieraufgaben wie das Arbeiten mit Objektsammlungen , das Durchführen von Berechnungen und das Kodieren/Dekodieren von Daten  unerlässlich sind.  Zusätzliche APIs sind in von der Community bereitgestellten Paketen verfügbar. Dieses Konzept ist dem aus .NET bekannten Bilbiothek Konzept sehr ähnlich.
Neben den Konzept sind auch die Inhalte in einigen Bibliotheken sehr ähnlich. So ähnelt die Bibliothek dart:async sehr dem .Net Namespace System.Threading.  Außerdem ist dart:Math sehr ähnlich wie System.MAth und dart.io zu System.IO.
Darüber hinaus können Sie Funktionen direkt in Dateien haben, ohne eine Klasse oder einen Namespace. Das ist fantastisch, wenn Sie funktionaler programmieren wollen. z. B. können Sie dies in eine Datei ganz allein stellen, nichts anderes wird benötigt.

\section{Asynchrone Benutzeroberfläche und Parallelität}

Das aufgabenbasierte asynchrone Programmiermodell stellt eine Abstraktion über asynchronen Code bereit. Der Quelltext kann dabei in gewohnter Weise als eine Folge von Anweisungen geschrieben werden und auch so gelesen werden, als ob jede Anweisung abgeschlossen wäre, bevor die nächste Anweisung beginnt. Der Compiler führt eine Reihe von Transformationen durch, da möglicherweise einige dieser Anweisungen gestartet werden und eine Task zurückgeben, die die derzeit ausgeführte Arbeit darstellt.
Ziel dieser Syntax ist es: Code zu aktivieren, der sich wie eine Folge von Anweisungen liest, aber in einer deutlich komplizierteren Reihenfolge ausgeführt wird, die auf einer externen Ressourcenzuordnung und dem Abschluss von Aufgaben basiert. Vergleichbar ist dies mit der Art und Weise, wie Menschen Anweisungen für Prozesse erteilen, die asynchrone Aufgaben enthalten.  \footcite[Vgl. ][Abgerufen am \today]{MicrosoftAsyncAwait2020} 

Die Verwendung von asychronen Methoden für lang laufende Aufgaben, wie das herunterladen von Daten, trägt dazu bei, dass die Benutzeroberfläche reaktionsfähig bleibt , während die Nichtverwendung dieser Methoden oder die unsachgemäße Verwendung dazu führen kann,  dass die Benutzeroberfläche Ihrer App nicht mehr auf Benutzereingaben reagiert, bis die lang laufende Aufgabe abgeschlossen ist.  Für beide in dieser Arbeit behandelten Frameworks gilt,  dass arbeitsaufwendige Aufgaben nicht auf in dem Thread durchgeführt werden sollten,  die für die Benutzeroberfläche zuständig ist, um ein einfrieren dieser zu verhindern. 
In Flutter werden die asynchronen Möglichkeiten, die die Sprache Dart bietet, auch \glq async\grq{} und \glq await\grq{}  genannt, um asynchrone Arbeiten auszuführen. Dies ist dem Vorgehen in \Csharp sehr ähnlich und ist der Lösung in Xamarin.Forms sehr ähnlich die ebenfalls die Schlüsselwörter  \glq async\grq{} und \glq await\grq{}  verwendet. 

\lstinputlisting[label={lst:DartAsync},caption={[Async und Await in Dart]{Async und Await in Dart\footcite[[In Anlehnung an ][Abgerufen am \today]{Pedley2019}}}, language=Dart]{SourceCode/DartAsyncAwait.Dart}

Um\footcitetext[Quelltext in Anlehnung an][Abgerufen am \today]{Pedley2019}  die Vorteile von mehreren Prozessorkernen nutzen zu können,  reicht dieses Konzept jedoch nicht aus,  in Flutter müssen potentielle Hintergrund-Thread manuell verwaltet werden,  ähnlich wie bei \glq async\grq{} und \glq await\grq{}  ist das vorgehen vergleichbar mit dem Vorgehen in \Csharp. So steht für den Start von rechenintensive Arbeiten sogenannte \glq Isolate\grq{} zur Verfügung,   dies ähnlich arbeiten wie Tasks in \Csharp. Dabei sind \glq Isolates \grq{}  sind separate Ausführungsthreads, die sich keinen Speicher mit dem Hauptspeicherheap der Ausführung teilen.  Dies ist ein Unterschied zu \glq Tasks\grq{} .  Das bedeutet, dass Sie vom Haupt-Thread aus nicht auf Variablen zugreifen oder Ihre Benutzeroberfläche durch den Aufruf von \glq IsetState()\grq{}  aktualisieren können.

\section{Netzwerkaufrufe}
Um Netzwerkaufrufe durchzuführen, um Daten von einem Server abzurufen oder Benutzereingaben des Anwenders zu übermittelt wird in Xamarin.Forms die Klasse HttpClient verwenden. Für die Arbeit in Flutter wird dafür das http-Paket verwendet, welches von einem großteil der Netzwerkfunktionalitäten abstrahiert und es einfach macht Netzwerkaufrufe zu tätigten.  Um eine Netzwerkanfrage zu stellen ist es wichtig, die vorher eingeführten Schlüsselwörter \glq async\grq{} und \glq await\grq{} zu verwenden, damit die Benutzeroberfläche auch während der Anfrage Reaktionsfähig bleibt.  Ein Netzwerkanfrage in Flutter wird in  \ref{lst:FlutterNetworkRequest} dargestellt.


\lstinputlisting[label={lst:FlutterNetworkRequest},caption={Flutter Network request}, language=Dart]{SourceCode/NetworkRequest.Dart}
\footcitetext[Quelltext in Anlehnung an][Abgerufen am \today]{Pedley2019} 


\section{Events}
Ein \footcitetext[Quelltext in Anlehnung an][Abgerufen am \today]{Pedley2019} Ereignis ist eine Meldung, die von einem Objekt gesendet wird, um das Auftreten einer Aktion zu signalisieren.  In dem vorherigen Kapitel wurde schon eingeführt, dass Xamarin.Forms ein ereignisgesteuertes Framework ist, welches Ereignisse wie den Klick auf eine Schaltfläche nutzen um Aktionen auszulösen.  Events können in \Csharp auch ohne \ac{xaml}-Dateien verwendet werden beispielsweise wenn eine  Programmlogik, z. B. das Ändern eines Eigenschaftswerts, ausgelöst. Das Objekt, von dem das Ereignis ausgelöst wird, wird als Ereignissender bezeichnet.  Dem Ereignissender ist nicht bekannt, welches Objekt oder welche Methode die ausgelösten Ereignisse empfangen (behandeln) wird.  Das Ereignis ist in der Regel ein Member des Ereignissenders.  Beispielsweise ist das Click-Ereignis ein Member der Klasse Button, und das PropertyChanged-Ereignis ist ein Member der Klasse, von der die INotifyPropertyChanged-Schnittstelle implementiert wird. \footcite[Vgl. ][Abgerufen am \today]{MicrosoftEvents2017} 

Anstelle von Ereignisse werden in Dart sogenannte Streams verwendet, die ähnlich wie Ereignisse arbeiten. 

Anstelle eines Ereignisses in \Csharp mit Delegaten, die dann alle aufgerufen werden, wenn ein Ereignis ausgelöst wird, arbeitet Dart in Streams.  Ein Stream ist ähnlich wie ein Ereignis, aber Sie öffnen ihn, hören ihn an und schließen ihn.

Der Vorteil dieses Ansatzes ist, dass Sie viele Dinge tun können, wie z. B. Werte transformieren oder Ereignisse für eine bestimmte Zeitspanne anhalten und vieles mehr. 

\lstinputlisting[label={lst:DartEvents},caption={[Events in Dart]{Events in Dart\footcite[[In Anlehnung an ][Abgerufen am \today]{Pedley2019}}}, language=Dart]{SourceCode/DartEvents.Dart}
 \footcitetext[Quelltext in Anlehnung an][Abgerufen am \today]{Pedley2019}

\section{Überführung einer beispielhaften Klassen}

