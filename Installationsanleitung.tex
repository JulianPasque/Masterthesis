\chapter{Anhang V: Installationsanleitung}
\label{chap:Installationsanleitung}

Für die Verwendung des Prototyps ist ein Windows Computer notwendig,  da er nur auf diesem Betriebssystem lauffähig ist.  
Auf dieser Plattform müssen verschiedene Installationen durchgeführt werden:

\begin{enumerate}
\setlength\itemsep{-0.6em}
	\item Visual Studio 2019: Beinhalten den Roslyn Compiler und können über die Webseite \url{https://visualstudio.microsoft.com/de/downloads/} heruntergeladen werden.  Über den Installer müssen anschließend die Komponenten für die Mobile und Desktop-Entwicklung mit .NET installiert werden.	
	\item Version 5.0 des .NET Framework
	\item Flutter-SDK:  Eine Installationsanleitung und die benötigten Ressourcen sind über die Webseite \url{https://flutter.dev/docs/get-started/install/windows} verfügbar.  Wichtig für die Ausführung des Compilers ist,  dass die Umgebungsvariable mit dem Namen \glq Path\grq{}  auf das \glqq flutter\textbackslash bin\grqq{} Verzeichnis verweist. 
	\item Android SDK-tools: Für eine Installation kann die Entwicklungsumgebung Android Studio über die Webseite \url{https://developer.android.com/studio} bezogen werden,  die alle benötigten Inhalte mitliefert. 
 
\end{enumerate}


Über die Kommandozeile können anschließend die Lizenzbedingungen der Android SDK gelesen und akzeptiert werden.  Dafür muss der Befehl \glqq flutter doctor \texttt{-{}-}android-licenses\grqq{} ausgeführt werden.  Mithilfe des  Befehls \glqq flutter -doctor\grqq{} kann die Installation überprüft werden.

Bevor der Compiler ausgeführt werden kann ist es notwendig die Xamarin.Forms Projektmappe in Visual Studio 2019 zu öffnen und über das Kontextmenü des Projektes alle Nuget-Packages wiederherzustellen.  Anschließend müssen alle Projekte in der Lösung gebaut werden.  So wird sichergestellt das alle Abhängigkeiten konstruiert und verfügbar sind. 

In einem letzten Schritt muss der Anwendung das Vertrauen ausgesprochen werden, damit sie auf dem Computer ausgeführt werden kann.
