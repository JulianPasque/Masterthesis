\pagenumbering{Roman}

\chapter*{Zusammenfassung}
\chaptermark{Zusammenfassung}
Die Einführung vom .NET Multi-platform App User Interface als Nachfolger von Xamarin.Forms zwingt Entwickler tiefgreifende Modifizierungen an bereits realisierten Anwendungen, die zukünftig nicht mehr von Aktualisierungen profitieren werden, vorzunehmen.

Die Abhängigkeit von Microsoft kann durch einen automatisierten Umstieg auf das von Google entwickelte Flutter Framework gelöst und die betroffenen Apps auf eine bewährte, leistungsfähigere Basis gestellt werden. In dieser Arbeit wurde analysiert, ob eine rein automatisierte Transformation möglich ist, oder manuelle Arbeitsschritte notwendig sind.  Zur Beantwortung der daraus resultierten Forschungsfrage wurde ein Source-To-Source Compiler entwickelt, der Projekte von Xamarin.Forms zu Flutter übersetzt. Vor der  Kompilierung wurden die Ansichten mit den dahinterstehenden Logiken herausgearbeitet und betrachtet.  Im Anschluss erfolgte der Austausch der visuellen Elemente durch Flutter Widgets, die zu diesem Zweck in Vorlagen mit Platzhaltern gespeichert waren. Nach der Umwandlung von den visuellen Eigenschaften konnten diese Platzhalter befüllt werden, um die Benutzeroberfläche zu finalisieren.  Im letzten Schritt konnten mithilfe des Roslyn Compilers, die in \Csharp{} geschriebene Anwendungslogik zu einem Syntaxbaum transformiert und nach dessen Durchlauf der Dart Quelltext konstruiert werden. Die ähnliche Syntax beider Sprachen erleichtert den Wechsel zwischen den Frameworks.  Unterschiede bei den verfügbaren Typen und Zugriffsmodifizierer bedurften jedoch einer genaueren Behandlung.

Zur Überprüfung,  ob der Prototyp das erwartete Ergebnis erzeugt, wurde eine speziell für diesen Zweck programmierte Xamarin.Forms App übersetzt und die erzeugte Flutter App mit der Ursprungsvariante verglichen. Das Resultat dieses Tests beweist, dass Xamarin.Forms Anwendungen ohne Funktionsverlust automatisiert und nur mit leichten Designabweichungen zu Flutter überführt werden können und potenziell nur wenige manuelle Nacharbeitungen erforderlich sind.  Einschränkungen des Prototyps, wie die Ausklammerung der Übersetzung des plattformspezifischen Quelltexts und des Styles der Anwendung, könnten zukünftig durch eine Weiterentwicklung als Open-Source Projekt realisiert und dann ebenfalls maschinell übersetzt werden.


\chapter*{Abstract}
\chaptermark{Abstract}

The introduction of the .NET Multi-platform App User Interface as the successor to Xamarin.Forms forces developers to make profound modifications to already realised applications that will no longer benefit from future updates.

The dependency on Microsoft can be solved by an automated switch to the Flutter Framework developed by Google and the affected apps can be placed on a proven, more powerful basis. In this thesis, it was analysed whether a purely automated transformation is possible or whether manual work steps are necessary.  To answer the resulting research question, a source-to-source compiler was developed to translate projects from Xamarin.Forms to Flutter. Before compilation, the views with the underlying logics were worked out and considered.  This was followed by replacing the visual elements with Flutter widgets, which were stored in templates with placeholders for this purpose. After the conversion of the visual properties, these placeholders could be filled in to finalise the user interface.  In the last step, the application logic written in \Csharp{} could be transformed into a syntax tree with the help of the Roslyn compiler, and the Dart source code could be constructed after it had been run through. The similar syntax of both languages makes it easier to switch between frameworks.  Differences in the available types and access modifiers, however, required more detailed treatment.

To check whether the prototype produces the expected result, a Xamarin.Forms app specially programmed for this purpose was translated and the generated Flutter app was compared with the original variant. The result of this test proves that Xamarin.Forms apps can be automated to Flutter without loss of functionality, with only slight design deviations, and potentially requiring little manual rework.  Limitations of the prototype, such as the exclusion of the translation of the platform-specific source code and the style of the application, could be realised in the future through further development as an open-source project and then also translated by machine.

