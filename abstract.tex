\pagenumbering{Roman}

\chapter*{Zusammenfassung}
\chaptermark{Zusammenfassung}
Durch die Einführung von .NET Multi-platform App User Interface müssen Xamarin.Forms Entwickler tiefgreifende Modifizierungen an bereits realisierten Anwendungen vornehmen, um zukünftig von Aktualisierungen zu profitieren.  Ein automatisierter Umstieg zu dem von Google entwickelten Flutter Framework würde die Änderungen verhindern und die bestehenden mobilen Anwendungen auf eine leistungsfähige und vermeidlich zukunftssichere Basis stellen.  Um diese Möglichkeit zu validieren wurde in dieser Arbeit analysiert,  ob diese automatisierte Transformation von bestehenden mobilen Anwendungen möglich ist,  und ob anschließend manuelle Arbeitsschritte notwendig sind.  Um diese Forschungsfrage beantworten zu können wurden in dieser Arbeit ein Source-To-Source Compiler entworfen,  und in Form eines Prototypen entwickelt.  

Als Grundlage für diesen Compiler war es notwendig,  die beiden Frameworks detailliert zu analysieren und ihre Arbeitsweisen zu vergleichen.  Dafür wurde eine Gegenüberstellung der Verfügbaren User-Interface Elemente durchgeführt und die Ansicht einer Anwendung überführt,  mit dem Ergebnis das der Austausch von Elementen ein valides vorgehen darstellt.  Die Syntax der beiden Programmiersprachen ist sehr ähnlich, was einen Umstieg erleichtert. Dennoch gibt es einige Unterschiede,  wie die Verfügbaren Typen und Modifizieren, die bei der Übersetzung berücksichtigt werden müssen.  

Anschließend wurde der Source-To-Source Compiler realisiert,  der in Kombination mit den Visual Studio Build Tools und der Flutter SDK die mobile Anwendung zu Flutter überführt. Um die Übersetzung zu validieren wurde ebenfalls eine mobile Anwendung mit Xamarin.Forms entwickelt und mithilfe des Compilers übersetzt und mit der ursprünglichen Variante verglichen.  
Dieser Vergleich zeigt,  dass Xamarin.Forms Anwendungen automatisiert zu Flutter überführt werden können und somit die Abhängigkeit zu dem von Microsoft entwickelten Framework lösen lassen.  Da der Compiler jedoch nur als Prototyp realisiert wurde,  wird plattformspezifischer Quelltext und der Style der Anwendung nicht übersetzt.  Um die zukünftige Weiterentwicklung zu garantieren soll der Compiler als Open-Source Projekt veröffentlicht werden. 
\chapter*{Abstract}
\chaptermark{Abstract}

Due to the introduction of .NET Multi-platform App User Interface,  Xamarin.Forms developers have to make profound modifications to already realised applications in order to benefit from future updates.  An automated switch to the Flutter Framework developed by Google would prevent the changes and put existing mobile apps on a powerful and avoidably future-proof footing.  In order to validate this possibility, this thesis analysed whether this automated transformation of existing mobile applications is possible and whether manual steps are necessary afterwards.  To answer this research question, a source-to-source compiler was designed and developed in the form of a prototype.  

As a basis for this compiler, it was necessary to analyse the two frameworks in detail and compare their working methods.  For this purpose, a comparison of the available user interface elements was carried out and the view of an application was transferred, with the result that the exchange of elements represents a valid procedure.  The syntax of the two programming languages is very similar, which facilitates a changeover. Nevertheless, there are some differences, such as the available types and modifications, which must be taken into account in the translation.  

Subsequently, the source-to-source compiler was realised, which, in combination with the Visual Studio Build Tools and the Flutter SDK, transfers the mobile application to Flutter. To validate the translation, a mobile application was also developed with Xamarin.Forms and translated using the compiler and compared with the original variant.  
This comparison shows that Xamarin.Forms applications can be automatically transferred to Flutter and thus solve the dependency on the framework developed by Microsoft.  However, since the compiler was only realised as a prototype, platform-specific source code and the style of the application are not translated.  In order to guarantee future development, the compiler will be published as an open-source project. 

