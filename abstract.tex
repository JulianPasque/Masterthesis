\pagenumbering{Roman}

\chapter*{Zusammenfassung}
\chaptermark{Zusammenfassung}
Die Einführung von .NET Multi-platform App User Interface zwingt Entwickler tiefgreifende 
Modifizierungen an bereits realisierten Xamarin.Forms Anwendungen vorzunehmen,  die zukünftig nicht von Aktualisierungen profitieren werden.  

Diese Abhängigkeit von Microsoft kann durch einen automatisierten Umstieg auf das von Google 
entwickelte Flutter Framework gelöst und die betroffenen Apps auf eine bewährte leistungsfähigere
Basis gestellt werden.  
In dieser Arbeit wurde analysiert, ob eine rein 
automatisierte Transformation von bestehenden mobilen Anwendungen möglich ist,  oder manuelle Arbeitsschritte notwendig sind.  Zur Beantwortung der daraus resultierten Forschungsfrage wurde ein Source-To-Source Compiler entwickelt,  der Projekte von Xamarin.Forms  zu Flutter übersetzt.

Für diesen Compilierungsvorgang wurde analysiert,  wie Ansichten und Anwendungslogiken in beiden Frameworks definiert und  transformiert werden können.  Visuelle Elemente können von Xamarin.Forms durch Flutter Wdigets ersetzt werden,  die zu diesem Zweck in Vorlagen mit Platzhaltern gespeichert wurden.  Nach einer Umwandlung der visuellen Eigenschaften konnten diese Platzhalter befüllt werden um die Benutzeroberfläche zu finalisieren.  
Durch den Roslyn Compiler konnte die in \Csharp{}  geschriebene Anwendungslogik zu einem Syntaxbaum umgewandelt werden der anschließend durchlaufen wurde um den Dart Quelltext zu konstruieren.  Die ähnliche Syntax beider Sprachen erleichtert einen Umstieg.   Unterschiede,  bei den verfügbaren Typen und Modifizierern,  bedurften  jedoch einer genaueren Behandlung.


Zur Überprüfung, ob der Prototyp das erwartete Ergebnis erzeugt, wurde eine speziell für diesen Zweck programmierte Xamarin.Forms App übersetzt und die erzeugte Flutter App mit der Ursprungsvariante verglichen.
Das Resultat dieses Tests beweist, dass Xamarin.Forms Anwendungen automatisiert, ohne 
Funktionsverlust, nur mit leichten Designabweichungen zu Flutter überführt werden können und nur wenige manuelle Nacharbeitung  erforderlich sind. 
Einschränkungen des Prototypens, wie die Ausklammerung der Übersetzung des 
plattformspezifischen Quelltexts und des Styles der Anwendung, könnten zukünftig durch eine 
Weiterentwicklung als Open-Source Projekt realisiert werden.

\chapter*{Abstract}
\chaptermark{Abstract}

Due to the introduction of .NET Multi-platform App User Interface,  Xamarin.Forms developers have to make profound modifications to already realised applications in order to benefit from future updates.  An automated switch to the Flutter Framework developed by Google would prevent the changes and put existing mobile apps on a powerful and avoidably future-proof footing.  In order to validate this possibility, this thesis analysed whether this automated transformation of existing mobile applications is possible and whether manual steps are necessary afterwards.  To answer this research question, a source-to-source compiler was designed and developed in the form of a prototype.  

As a basis for this compiler, it was necessary to analyse the two frameworks in detail and compare their working methods.  For this purpose, a comparison of the available user interface elements was carried out and the view of an application was transferred, with the result that the exchange of elements represents a valid procedure.  The syntax of the two programming languages is very similar, which facilitates a changeover. Nevertheless, there are some differences, such as the available types and modifications, which must be taken into account in the translation.  

Subsequently, the source-to-source compiler was realised, which, in combination with the Visual Studio Build Tools and the Flutter SDK, transfers the mobile application to Flutter. To validate the translation, a mobile application was also developed with Xamarin.Forms and translated using the compiler and compared with the original variant.  
This comparison shows that Xamarin.Forms applications can be automatically transferred to Flutter and thus solve the dependency on the framework developed by Microsoft.  However, since the compiler was only realised as a prototype, platform-specific source code and the style of the application are not translated.  In order to guarantee future development, the compiler will be published as an open-source project. 

