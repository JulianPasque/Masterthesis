\pagenumbering{Roman}

\chapter*{Zusammenfassung}
\chaptermark{Zusammenfassung}
Die Einführung vom .NET Multi-platform App User Interface als Nachfolger von Xamarin.Forms zwingt Entwickler tiefgreifende Modifizierungen an bereits realisierten Anwendungen, die zukünftig nicht mehr von Aktualisierungen profitieren werden, vorzunehmen.

Die Abhängigkeit von Microsoft kann durch einen automatisierten Umstieg auf das von Google entwickelte Flutter Framework gelöst und die betroffenen Apps auf eine bewährte, leistungsfähigere Basis gestellt werden. In dieser Arbeit wurde analysiert, ob eine rein automatisierte Transformation möglich ist, oder manuelle Arbeitsschritte notwendig sind.  Zur Beantwortung der daraus resultierten Forschungsfrage wurde ein Source-To-Source Compiler entwickelt, der Projekte von Xamarin.Forms zu Flutter übersetzt. Vor der  Kompilierung wurden die Ansichten mit den dahinterstehenden Logiken herausgearbeitet und betrachtet.  Im Anschluss erfolgte der Austausch der visuellen Elemente durch Flutter Widgets, die zu diesem Zweck in Vorlagen mit Platzhaltern gespeichert waren. Nach der Umwandlung von den visuellen Eigenschaften konnten diese Platzhalter befüllt werden, um die Benutzeroberfläche zu finalisieren.  Im letzten Schritt konnten mithilfe des Roslyn Compilers, die in \Csharp{} geschriebene Anwendungslogik zu einem Syntaxbaum transformiert und nach dessen Durchlauf der Dart Quelltext konstruiert werden. Die ähnliche Syntax beider Sprachen erleichtert den Wechsel zwischen den Frameworks.  Unterschiede bei den verfügbaren Typen und Modifizieren bedurften jedoch einer genaueren Behandlung.

Zur Überprüfung,  ob der Prototyp das erwartete Ergebnis erzeugt, wurde eine speziell für diesen Zweck programmierte Xamarin.Forms App übersetzt und die erzeugte Flutter App mit der Ursprungsvariante verglichen. Das Resultat dieses Tests beweist, dass Xamarin.Forms Anwendungen ohne Funktionsverlust automatisiert und nur mit leichten Designabweichungen zu Flutter überführt werden können und potenziell nur wenige manuelle Nacharbeitungen erforderlich sind.  Einschränkungen des Prototyps, wie die Ausklammerung der Übersetzung des plattformspezifischen Quelltexts und des Styles der Anwendung, könnten zukünftig durch eine Weiterentwicklung als Open-Source Projekt realisiert und dann ebenfalls maschinell übersetzt werden.


\chapter*{Abstract}
\chaptermark{Abstract}

The introduction of .NET Multi-platform App User Interface forces developers to make profound modifications to already realised Xamarin.Forms applications that will not benefit from future updates. 

The dependency on Microsoft can be resolved by an automated switch to the Flutter Framework developed by Google and put the affected apps on a more on a proven basis with better performance. In this thesis it has been analysed whether a purely automated transformation is possible or whether manual steps are necessary.  To answer the resulting research question, a source-to-source compiler was developed that translates projects from Xamarin.Forms to Flutter. The first step of the compilation was the analysis of views and their underlying logics.  Visual elements of Xamarin.Forms were then replaced by Flutter widgets, which for this purpose were stored in templates with placeholders. After a conversion of the visual properties, these placeholders could be filled in in order to finalise the user interface. 
In the last step, with the help of the Roslyn compiler, the application logic written in \Csharp{} was transformed into a syntax tree and after running through it, the Dart source code could be constructed. The similar syntax of both languages makes it easier to switch between the frameworks.  Differences in the available types and modifiers, however, required more detailed treatment.

In order to check whether the prototype produces the expected result, a specially Xamarin.Forms app specially programmed for this purpose was translated and the generated flutter app app was compared with the original version. The result of this test proves that Xamarin.Forms applications can be transferred to Flutter automatically, without loss of function, with only slight design deviations. . Limitations of the prototype, such as the exclusion of the the translation of the platform-specific source code and the style of the application, could be be realised in the future through further development as an open-source Project.

