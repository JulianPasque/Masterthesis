\chapter{Programmiersprachliche Grundbegriffe}
In diesem Kapitel werden die notwendigen Terminologien im Hinblick auf Compilerbau-Aspekte eingeführt.
\section{ Programm}
Programme können in unterschiedlichen Repräsentationen vorkommen für ein Einheitliches Verständnis ist es notwendig, zu wissen, von welcher Präsentation die Rede ist. In der Literatur werden die folgenden Repräsentationen unterschieden
\begin{itemize}
\item Der Quelltext:
\item Ein Objektmodul:
\item Ein ausführbares Programm:
\item Ein Prozess:
\end{itemize}

\section{ Syntax und Semantik}
Die Syntax einer Programmiersprache ist das Aussehen bzw. die Struktur des Quelltextes. Der komplette Quelltext eines Programms muss syntaktisch korrekt sein, damit er übersetzt werden kann.  \\
Die Semantik ist die Bedeutung, die den verschiedenen syntaktischen Strukturen zugeordnet werden kann. Es ist also von der Semantik abhängig auf welche Art und Weise die Weiterverabeitung durch den Compiler erfolgt. 
