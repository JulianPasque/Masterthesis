\chapter{Einleitung}
Die Entwicklung von verschiedenen mobilen Geräten mit unterschiedlichsten Hardwarekomponenten und Betriebssystemen hat einen stark fragmentierten Markt ergeben.\footcite[Vgl.][S. 3]{Joorabchi2016}  Diese Situation hat einen direkten Einfluss auf die Softwareentwicklung, da die dedizierte Programmierung für die einzelnen Plattformen ressourcenintensiv ist.  Durch Realisierung von Web- und  Hybriden Apps können Softwareprojekte von der darunterliegenden Plattform abstrahieren und plattformübergreifend verwendet werden.  Diese Anwendungen haben jedoch, wie schon ausführlich im wissenschaftlichen Diskurs ausgeführt,  eine schlechtere Performance und nur begrenzt Zugriff auf die plattformspezifischen Funktionalitäten. 

Durch die Kombination der Vorteile von Web- und Hybriden- mit denen von Nativen Anwendungen konnten Frameworks wie Xamarin.Forms und Flutter Programmierern die Möglichkeit bieten Ihre Anwendungen auf mehreren Plattformen bereit zu stellen.  Diese Apps haben neben einer guten Performance auch Zugriff auf sämtliche plattformspezifischen Funktionalitäten.  Durch die Abstraktion von Hardware und Betriebssystem können Apps mit einer gemeinsamen Quelltextbasis und somit mit geringeren Ressourcenaufwand entwickelt werden. \footcite[Vgl.][S. 295]{Vollmer2017} 

Der Möglichkeit Ressourcen zu sparen steht das Risiko der Abhängigkeit gegenüber, da sich die oben genannten Frameworks zur Cross-Plattform-Entwicklung in den verwendeten Programmiersprachen sowie ihrer Arbeitsweise grundlegend unterscheiden.  Ein Wechsel zwischen den einzelnen Alternativen ist daher mit enormen Arbeitsaufwänden verbunden. \footcite[Vgl.][S. 64]{Wissel2017}  

\section{Motivation}
Im Mai 2020 hat Microsoft mit .NET MAUI einen Nachfolger für das Xamarin.Forms Framework angekündigt, der im Herbst 2021 zusammen mit der sechsten Version des .NET Frameworks veröffentlicht werden soll. Zum aktuellen Zeitpunkt ist bereits berkannt, dass der Umstieg grundlegende Änderung mit sich bringt und Anwendungen die mit Hilfe von Xamarin.Forms entwickelt wurden angepasst werden müssen.\footcite[Vgl.][Abgerufen am 28.10.2020]{Hunter2020}

Für Xamarin.Forms Entwickler wird es also unausweichlich sein, grundlegende Änderungen an bereits realisierten Anwendungen vorzunehmen,  um in der Zukunft von Aktualisierungen zu profitieren.  Unternehmen und einzelne Programmierer stehen vor der Entscheidung,  ob ein Umstieg auf das leistungsfähige Flutter sinnvoller ist,  als die Anpassungen für das neue noch nicht erprobte .NET MAUI, das von einer Firma federführend entwickelt wird,  welche leichtfertig mit der Abhängigkeit von Entwicklern umgeht.
Ein automatisierter Umstieg auf das von Google entwickelte Framework Flutter würde also nicht nur die Anpassungen an .NET MAUI vermeiden, sondern die App auf eine vermeintlich zukunftssichere Basis stellen.  

\section{Ziel der Arbeit}
Im Rahmen dieser Arbeit soll ein Source-To-Source Compiler zwischen den Frameworks Xamarin.Forms und Flutter realisiert werden, mit dessen Hilfe die folgende zentrale Forschungsfrage beantwortet werden soll: "Können Apps komplett automatisiert von Xamarin.Forms zu Flutter übersetzt werden, oder sind manuelle Arbeitsschritte erforderlich?"


\section{Gliederrung}
