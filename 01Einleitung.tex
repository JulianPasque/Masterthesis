\chapter{Einleitung}
Die Entwicklung von verschiedenen mobilen Geräten mit unterschiedlichsten Hardwarekomponenten und Betriebssystemen hat einen stark fragmentierten Markt ergeben.\footcite[Vgl.][S. 3]{Joorabchi2016}  Diese Situation hat einen direkten Einfluss auf die Softwareentwicklung, da die dedizierte Programmierung für die einzelnen Plattformen ressourcenintensiv ist.  Durch Realisierung von Web- und hybriden Apps können Softwareprojekte von der darunterliegenden Plattform abstrahieren und plattformübergreifend verwendet werden.  Diese Anwendungen haben jedoch, wie schon ausführlich im wissenschaftlichen Diskurs ausgeführt,  eine schlechtere Performance und nur begrenzt Zugriff auf die plattformspezifischen Funktionalitäten.  \footcite[Vgl.][S. 110ff.]{Barton2016} 

Durch die Kombination der Vorteile von Web- und hybriden Anwendungen mit denen von nativen konnten Frameworks wie Xamarin.Forms und Flutter Programmierern die Möglichkeit bieten,  ihre Anwendungen auf mehreren Plattformen bereit zu stellen.  Diese Apps haben neben einer guten Performance auch Zugriff auf sämtliche plattformspezifischen Funktionalitäten.  Durch die Abstraktion von Hardware und Betriebssystem können Apps mit einer gemeinsamen Quelltextbasis und somit mit geringerem Ressourcenaufwand entwickelt werden. \footcite[Vgl.][S. 295]{Vollmer2017} 

Der Möglichkeit, Ressourcen zu sparen, steht das Risiko der Abhängigkeit gegenüber, da sich die oben genannten Frameworks zur Cross-Plattform-Entwicklung in den verwendeten Programmiersprachen sowie ihrer Arbeitsweise grundlegend unterscheiden.  Ein Wechsel zwischen den einzelnen Alternativen ist daher mit enormen Arbeitsaufwänden verbunden. \footcite[Vgl.][S. 64]{Wissel2017}  

\section{Motivation}
Im Mai 2020 hat Microsoft mit dem Multi-platform App User Interface (.NET MAUI) einen Nachfolger für das Xamarin.Forms Framework angekündigt, der im Herbst 2021 zusammen mit der sechsten Hauptversion des .NET Frameworks veröffentlicht werden soll. Zum aktuellen Zeitpunkt ist bereits bekannt, dass der Umstieg grundlegende Änderungen mit sich bringt und Anwendungen,  die mit Hilfe von Xamarin.Forms entwickelt wurden,  angepasst werden müssen.\footcite[Vgl.][Abgerufen am 28.10.2020]{Hunter2020}

Für Xamarin.Forms Entwickler wird es also unausweichlich sein,  tiefgreifende Modifizierungen an bereits realisierten Anwendungen vorzunehmen,  um in der Zukunft von Aktualisierungen zu profitieren.  Unternehmen und einzelne Programmierer stehen vor der Entscheidung,  ob ein Umstieg auf das leistungsfähige Flutter sinnvoller ist,  als die Anpassungen für das neue noch nicht erprobte .NET MAUI,  das federführend von einer Firma entwickelt wird,  welche leichtfertig mit der Abhängigkeit von Entwicklern umgeht.

Ein automatisierter Umstieg auf das von Google entwickelte Framework Flutter würde also nicht nur die Anpassungen an .NET MAUI vermeiden,  sondern die mobile Anwendung auf eine vermeintlich zukunftssichere Basis stellen.  Denn obwohl Google in der Vergangenheit schon manche Projekte eingestellt hat,  wie zum Beispiel Google Nexus oder Google Hangouts,  ist damit bei Flutter aufgrund des Erfolges nicht zu rechnen.  Nach offizieller Aussage von Tim Sneath, dem Produkt Manager des Frameworks,  haben im Jahr 2020 mehr als zwei Millionen Entwickler Flutter verwendet und über 50.000 mobile Anwendungen programmiert. \footcite[Vgl.][Abgerufen am 28.10.2020]{Sneath2020} Neben der hohen Verbreitung des Frameworks,  konnte das Portal Stackoverflow in seinen jährlichen Umfragen auch eine hohe Beliebtheit unter Softwareentwicklern in den Jahren 2019\footcite[Vgl.][Abgerufen am 28.10.2020]{Stack2019} und 2020\footcite[Vgl.][Abgerufen am 28.10.2020]{Stack2020} ermitteln.  Im März 2021 hat Flutter darüber hinaus die zweite Hauptversion von Flutter veröffentlicht,  welche zusätzlich Support für die Entwicklung von Webseiten zur Verfügung stellt. \footcite[Vgl.][Abgerufen am 28.10.2020]{GoogleFlutter2}


\section{Ziel der Arbeit}
Im Rahmen dieser Arbeit soll ein Source-To-Source Compiler zwischen den Frameworks Xamarin.Forms und Flutter realisiert werden, mit dessen Hilfe die folgende zentrale Forschungsfrage beantwortet werden soll: "Können Apps komplett automatisiert von Xamarin.Forms zu Flutter übersetzt werden, oder sind manuelle Arbeitsschritte erforderlich?"

\section{Gliederung}
Um diese Forschunsgfrage beantworten zu können, wird in Kapitel \ref{chap:Compiler} auf die theoretischen Grundlagen von Software-Übersetzern eingegangen.  Anschließend wird in Kapitel \ref{chap:CompilerEntwurf} auf den Entwurf des in dieser Arbeit zu implementierenden Compiler eingegangen, bevor in Kapitel \ref{chap:CrossPlattformFrameworks} die Unterschiede zwischen den Frameworks Xamarin.Forms und Flutter behandelt werden.  Darauf aufbauend wird in Kapitel \ref{chap:Realisierung} der Source-To-Source Compiler realisiert und in dem darauf folgen Kapitel \ref{chap:Qualitätssicherung} getestet bevor in Kapitel \ref{chap:FazitAusblick} die Forschungsfrage beantwortet wird und ein Fazit gezogen wird. 