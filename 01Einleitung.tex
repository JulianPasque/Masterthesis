%%% bitte die nächsten 5 Zeilen löschen
\chapter{Einleitung}
Durch die Entwicklung von verschiedenen mobilen Geräten mit unterschiedlichsten Hardwarekomponenten und Betriebssystemen hat sich ein stark fragmentierter Markt ergeben.\footcite[Vgl.][S. 3]{Joorabchi2016}  Diese Marktentwicklung hat einen direkten Einfluss auf die Softwareentwicklung für mobile Endgeräte und damit zu der Entwicklung von Cross-Plattform Frameworks wie Xamarin.Forms und Flutter gesorgt. Durch die Abstraktion von Hardware und Betriebssystem bieten diese Frameworks die Möglichkeit Anwendungen für die verschiedenen Plattformen mit einer gemeinsamen Quelltextbasis zu entwickeln und somit Kosten und Zeit zu sparen. \footcite[Vgl.][S. 295]{Vollmer2017} 
Der Möglichkeit Entwicklungsressourcen zu sparen steht das Risiko der Abhängigkeit entgegen, da sich alle Frameworks zur Cross-Plattform-Programmierung in den verwendeten Programmiersprachen sowie ihrer Arbeitsweise unterscheiden, ist ein Wechsel zwischen den einzelnen alternativen mit enormen Arbeitsaufwenden verbunden. \footcite[Vgl.][S. 64]{Wissel2017}  
\section{Ziel der Arbeit}
Im Rahmen dieser Arbeit soll ein Source-To-Source Compiler zwischen den Frameworks Xamarin.Forms und Flutter realisiert werden mit dessen Hilfe die folgende zentrale Forschungsfrage beantwortet werden soll: "Können Apps komplett automatisiert von Xamarin.Forms zu Flutter übersetzt werden, oder sind manuelle Arbeitsschritte erforderlich?"
\section{Motivation}
Im Mai 2020 hat Microsoft mit .NET MAUI einen Nachfolger für das Xamarin.Forms Framework angekündigt, der im Herbst 2021 zusammen mit der sechsten Version des .NET Frameworks veröffentlicht werden soll. Zum aktuellen Zeitpunkt ist bereits angekündigt, dass .NET MAUI grundlegende Änderung mit sich bringt und Anwendungen die mit Hilfe von Xamarin.Forms entwickelt wurden angepasst werden müssen.\footcite[Vgl.][Abgerufen am 28.10.2020]{Hunter2020}

Für Xamarin.Forms Entwickler wird es also unausweichlich sein, grundlegende Änderungen an bereits realisierten Anwendungen vorzunehmen um in der Zukunft von Aktualisierungen zu profitieren. Aus diesem Grund eignet sich die Zeit, um auch alternative Frameworks zu analysieren und zu überprüfen ob ein  Umstieg lohnenswert ist. Das Flutter Framework erfreut sich unter Entwicklern einer wachsenden Beliebtheit und ist nicht nur auf Grund seiner performanten Anwendungen mittlerweile weit verbreitet. 

Ein automatisierter Umstieg auf das von Google entwickelte Framework würde also nicht nur die Anpassungen an .NET Maui verhindern, sondern auch leistungsfähigere Anwendungen generieren. 

\section{Gliederrung}
