\chapter{Fazit}
\label{chap:FazitAusblick}
Ein Source-To-Source Compiler,  wie in dieser Arbeit entwickelt, stellt einen Lösungsansatz für Xamarin.Forms App-Entwickler dar,  die sich nach dem Ende des Supports vom Anbieter Microsoft trennen und trotzdem eine bestehende Anwendung erhalten möchten.  Das generelle Problem der Abhängigkeit ist jedoch nicht durch eine Abwendung von Microsoft gelöst, da das von der Firma Google entwickelte Flutter Framework ebenfalls, wenn auch derzeit nicht absehbar,  eingestellt werden könnte.  Die Idee,  Programmierern durch automatisierte Übersetzungen eine Wahlfreiheit und Wechseloptionen zu ermöglichen, rechtfertigt jedoch eine Weiterentwicklung des Prototypen.

\section{Open Source Veröffentlichung}

Der Umfang des Prototypen wurde durch die in Kapitel \ref{chap:CompilerEntwurf} ausgeschlossenen Aspekte reduziert,  da sie für die Beantwortung der Forschungsfrage eine untergeordnete Rolle spielen.  Diese Einschränkung besteht bei den meisten Xamarin.Forms Apps nicht.  
Eine Veröffentlichung des Compilers als Open Source Projekt könnte helfen,  komplexere mobile Anwendungen zu übersetzen,  indem Programmierer den dann für die Allgemeinheit verfügbaren Quelltext auf spezielle Gegebenheiten anpassen und wiederum publizieren können.  Unternehmen wie DevExpress,  Syncfusion und Telerik bieten sowohl Steuerelemente für 
Xamarin.Forms,  als auch Widgets für Flutter und hätten die Möglichkeit,  ihre eigenen \ac{ui}-Elemente 
durch den Compiler ersetzen zu lassen und somit dessen Umfang zu erhöhen.  Die Open Source Lizenz GPL 3.0 würde garantieren,  dass Veränderungen am Source-To-Source Compiler wiederum quelloffen sein müssen.  Eine solche Veröffentlichung könnte also die Compilerentwicklung hinsichtlich Geschwindigkeit und Zielgenauigkeit erhöht werden.

\section{Beantwortung der Forschungsfrage}
Es besteht die Möglichkeit,  Apps von Xamarin.Forms zu Flutter vollautomatisiert ohne manuelle 
Arbeitsschritte zu übersetzten.  Der Source-To-Source Compiler überträgt die Xamarin.Forms Testapp lückenlos und zeigt damit die Machbarkeit einer maschinellen Transformation.   Damit jedoch auch produktive Anwendungen vollständig, inklusive plattformspezifischer Implementierungen oder anwendungsspezifischer Erweiterungen, ohne manuelles Zutun übersetzt werden können,  sind 
Weiterentwicklungen beispielsweise durch eine Open-Source Veröffentlichung nötig.
Im aktuellen Zustand ist der Compiler schon in der Lage,  komplexe visuelle Darstellungen
und Quelltexte zu übersetzen und somit den Entwicklungsaufwand bei einem Umstieg auf Flutter 
merklich zu reduzieren.
Kritisch angemerkt sei,  dass derzeit noch manuelle Arbeitsschritte bei der Übersetzung notwendig werden,  sobald einer der ausgeschlossenen Aspekte verwendet wird.


\section{Ausblick}
Bereits für Ende 2021 ist das Supportende für Xamarin.Forms beschlossen und die Einführung von 
.NET \ac{maui} angekündigt.  Eine vollständige Funktionsfähigkeit des Compilers wäre noch in diesem Jahr wünschenswert,  um Unternehmen und Programmierern eine rechtzeitige Alternative zu bieten. 

Anwendungsentwickler bekommen durch den Umstieg von Xamarin.Forms zu Flutter seit März 2021 die Möglichkeit,  ihre Anwendung als Webseite zu veröffentlichen und somit eine weitere Plattform zu nutzen.  Jedoch sind zusätzliche Implementierungen in der Flutter App notwendig,  um die Webdarstellung auf großen Displays zu optimieren.  Flutter treibt seine Frameworkentwicklung durch Neuerungen wie den Support der Webnutzung weiter voran,  sodass derzeit nicht der 
Eindruck entsteht, dass ein Supportende in naher Zukunft zu befürchten ist. 
Ob die Einführung von .Net \ac{maui} wegweisende Innovationen mit sich bringt,  ist bisher genau so unbekannt wie der Programmieraufwand für den Umstieg von Xamarin.Forms.  Der Source-To-Source Compiler stellt mit der automatisierten Möglichkeit,  Xamarin.Forms zu Flutter zu übersetzen,  eine berechenbare Alternative zu .NET \ac{maui} bereit.