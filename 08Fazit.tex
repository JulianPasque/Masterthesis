\chapter{Fazit und Ausblick}
\label{chap:FazitAusblick}
Ein Source-To-Source Compiler,  wie in dieser Arbeit angedacht, stellt einen Lösungsansatz für Xamarin.Forms App-Entwickler dar,  die sich nach dem Ende des Supports vom Anbieter Microsoft trennen und trotzdem eine bestehende Anwendung erhalten möchten.  Das generelle Problem der Abhängigkeit ist jedoch nicht durch eine Abwendung von Microsoft gelöst, da das von der Firma Google entwickelte Flutter Framework ebenfalls, wenn auch derzeit nicht absehbar,  eingestellt werden kann. Der Ausbau der Idee,  durch grundsätzlich automatisierte Übersetzungen Programmierern eine Wahlfreiheit und Wechseloptionen zu ermöglichen rechtfertigt eine Weiterentwicklung des Prototypen.

\section{Open Source Veröffentlichung}

Der Umfang des Prototypen konnte durch die in Kapitel \ref{chap:CompilerEntwurf}  ausgeschlossenen Elemente 
reduziert werden, diese Präklusion besteht bei den meisten Xamerin.Forms Apps nicht. 
Eine Veröffentlichung des Compilers als Open Source Projekt könnte helfen, diese komplexeren 
mobilen Anwendungen zu übersetzen,  indem Programmierer den für die Allgemeinheit verfügbaren 
Quelltext auf ihre speziellen Gegebenheiten anpassen und wiederum publizieren.
Unternehmen wie DevExpress,  Syncfusion und Telerik bieten sowohl Steuerelemente für 
Xamarin.Forms, als auch Widgets für Flutter und hätten die Möglichkeit ihre eigenen \ac{ui}-Elemente 
durch den Compiler ersetzten zu lassen und somit dessen Umfang zu erhöhen.  
Durch Open Source Veröffentlichung könnte die Entwicklung in Geschwindigkeit und Zielgenauigkeit 
erhöht werden.

\section{Beantwortung der Forschungsfrage}
Der Source -To-Source Compiler übersetzt die Xamarin.Forms Testapp und zeigt damit die 
Machbarkeit einer zumindest teilautomatisierten Übersetzung zu Flutter.
Zur Kompilierung von produktiven Anwendungen ist eine Weiterentwicklung nötig, sodass 
plattformspezifische Implementierungen und viele Erweiterungen für das Xamarin.Forms Framework 
ebenfalls umgewandelt werden können.
Im aktuellen Zustand ist der Compiler schon in der Lage komplexe visuelle Darstellungen
und Quelltexte zu übersetzen und somit den Entwicklungsaufwand bei einem Umstieg auf Flutter 
merklich zu reduzieren.
Da bereits für Ende 2021 das Supportende für Xamarin.Forms beschlossen und die Einführung von 
MAUI angekündigt wurde, wäre eine vollständige Funktionsfähigkeit des Compilers noch in diesem 
Jahr nötig, um Unternehmen und Programmierern eine geeignete Alternative zu bieten. 

\section{Notwendige manuelle Arbeitsschritte}

In der aktuellen Form sind noch manuelle Arbeitsschritte für die Transformation von produktiven mobilen Anwendungen notwendig.  Da der Prototyp keinen plattformspezifischen Quelltext übersetzt müssen diese mit Hilfe der Plattformchannel nachträglich realisiert werden.  Ebenso müssen die Splashscreens überführt werden falls der während der Übersetzung generierte nicht ausreichend ist.  Da auch der Style der Anwendung nicht übernommen wurde ist eine visuelle Differenz zwischen den Anwendungen zu erwarten, daher ist eine nachträgliche Anpassung des visuellen Erscheinungsbildes wahrscheinlich. 

Der Prototyp wurde ausschließlich mit Hilfe des Testobjektes getestet,  die nicht alle Steuerelemente,  Layouts und Kombinationen derer abbildet.  Außerdem implementiert diese nicht alle Features der Programmiersprache \Csharp{}.  Bei der Übersetzung anderer Anwendung ist nach der Übersetzung ein vollständiger Regressionstest notwendig, um mögliche Seiteneffekte auszuschließen. 