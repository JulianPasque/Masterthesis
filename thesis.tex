% -*- coding: utf-8 -*-
\documentclass{scrbook} % Entspricht den Typografieregeln eines Buches
\usepackage{wbh-thesis} % Laden der Vorlage

\title{Realisierung eines Source-to-Source Compilers zwischen Xamarin.Forms und Flutter zur automatisierten Transformation bestehender mobiler Anwendungen}
\Autor[m]{Julian Pasqué} % Option entspricht dem Genus für Autor/in bzw. 
                            % Studierende/r:  m = Autor bzw. Studierender
                            %                 f = Autorin bzw. Studierende
\Fachbereich{INF} % Entspricht einer der folgenden Optionen für den Fachbereich:
                  % * ING = Ingenieurwissenschaften
                  % * INF = Informatik
                  % * WRT = Wirtschaftsingeniuerwesen und Technologiemanagement
\Betreuung[m]{Dr. Thomas Kalbe} % Option entspricht dem Genus für 
                                    % Betreuer/in:  m = Betreuer
                                    %               f = Betreuerin
\Matrikelnummer{902953}
\Abgabetermin{1.\,April~2021}
\Anschrift{Römlinghovener Str. 33}{53227 Bonn}

\begin{document}
  \maketitle          % Setzt die Titelseite
  \frontmatter        % Eigenschaften für den Vorspann setzen
  \Abstract{abstract} % Zusammenfassung (Abstract) aus Datei
  \tableofcontents    % Setzt das Inhaltsverzeichnis
  \mainmatter         % Eigenschaften für den Hauptteil setzen
  
%%% bitte die nächsten 5 Zeilen löschen
\chapter{Testkapitel}
Der eigentliche Inhalt folgt an dieser Stelle.

\section{Testabschnitt}
Und hier ein kurzer Text mit einem Quellenverweis \autocite{Schlosser14}.

  % \include{kap_einleitung}
  % \include{kap_analyse}
  % \include{kap_entwurf}
  % \include{kap_realisierung}
  % \include{kap_resuemee}
  
  \appendix                       % Leitet den Nachspann ein
  \listoftables                   % Setzt das Tabellenverzeichnis
  \listoffigures                  % Setzt das Abbildungsverzeichnis
  \printbibliography              % Setzt das Literaturverzeichnis
  \printEidesstattlicheErklaerung % Setzt die eidesstattlichen Erklärung
\end{document}
