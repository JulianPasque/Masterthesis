% -*- coding: utf-8 -*-
\documentclass{scrbook}[headings=normal]% Entspricht den Typografieregeln eines Buches
\usepackage{wbh-thesis} % Laden der Vorlage

\usepackage[automark,headsepline,plainheadsepline]{scrlayer-scrpage}
\clearpairofpagestyles % <==============================================
\ihead*{\leftmark} % <==================================================
\ohead*{\pagemark} % <==================================================
\pagestyle{scrheadings}

\RedeclareSectionCommand[
  afterindent=false,
  beforeskip=0bp,
  afterskip=0bp]{chapter}
\RedeclareSectionCommand[
  afterindent=false,
  beforeskip=12bp,
  afterskip=6bp]{section}
\RedeclareSectionCommand[
  afterindent=false,
  beforeskip=12bp,
  afterskip=6bp]{subsection}
\RedeclareSectionCommand[
  afterindent=false,
  beforeskip=6bp,
  afterskip=6bp]{subsubsection}
\RedeclareSectionCommand[
  runin=true,
  afterskip=1em]{paragraph}
\RedeclareSectionCommand[
  runin=true,
  afterskip=1em]{subparagraph}

\begin{document}
  \begin{titlepage}
\begin{center}
\newcommand{\HRule}{\rule{.9\linewidth}{.6pt}} % New command to make the lines in the title page

\vspace*{.06\textheight}
{\scshape\LARGE Wilhelm Büchner Hochschule}\vspace{1.5cm} % University name
\textsc{\Large Masterthesis}\\[0.5cm] % Thesis type

\HRule \\[0.4cm] % Horizontal line
{\huge \bfseries Realisierung eines Source-to-Source Compilers zwischen Xamarin.Forms und Flutter zur automatisierten Transformation bestehender mobiler Anwendungen}\vspace{0.4cm} % Thesis title
\HRule \\[1.5cm] % Horizontal line
 
\begin{minipage}[t]{0.4\textwidth}
\begin{flushleft} \large
\emph{Author:}\\
Julian Pasqué % Author name - remove the \href bracket to remove the link
\end{flushleft}
\end{minipage}
\begin{minipage}[t]{0.4\textwidth}
\begin{flushright} \large
\emph{Betreuer:} \\
Dr. Thomas Kalbe % Supervisor name - remove the \href bracket to remove the link  
\end{flushright}
\end{minipage}\\[3cm]
 
\vfill


\large Masterstudiengang  Verteilte und mobile Anwendungen\\[0.8cm] % Research group name and department name
\large Fachbereich Informatik\\[0.8cm] % Research group name and department name
\large Matrikelnummer: 902953 \\[0.8cm] % Research group name and department name
 
\vfill

{\large \today}\\[6cm] % Date
 
\vfill
\end{center}
\end{titlepage}  % Setzt die Titelseite
  \frontmatter        % Eigenschaften für den Vorspann setzen
  \include{Abstract}  % Setzt die Titelseite
  \tableofcontents    % Setzt das Inhaltsverzeichnis
  \listoffigures      % Setzt das Abbildungsverzeichnis
  \listoftables       % Setzt das Tabellenverzeichnis
  \mainmatter         % Eigenschaften für den Hauptteil setzen
  





  %%% bitte die nächsten 5 Zeilen löschen
\chapter{Einleitung}
Durch die Entwicklung von verschiedenen mobilen Geräten mit unterschiedlichsten Hardwarekomponenten und Betriebssystemen hat sich ein stark fragmentierter Markt ergeben.\footcite[Vgl.][S. 3]{Joorabchi2016}  Diese Marktentwicklung hat einen direkten Einfluss auf die Softwareentwicklung für mobile Endgeräte und damit zu der Entwicklung von Cross-Plattform Frameworks wie Xamarin.Forms und Flutter gesorgt. Durch die Abstraktion von Hardware und Betriebssystem bieten diese Frameworks die Möglichkeit Anwendungen für die verschiedenen Plattformen mit einer gemeinsamen Quelltextbasis zu entwickeln und somit Kosten und Zeit zu sparen. \footcite[Vgl.][S. 295]{Vollmer2017} 
Der Möglichkeit Entwicklungsressourcen zu sparen steht das Risiko der Abhängigkeit entgegen, da sich alle Frameworks zur Cross-Plattform-Programmierung in den verwendeten Programmiersprachen sowie ihrer Arbeitsweise unterscheiden, ist ein Wechsel zwischen den einzelnen alternativen mit enormen Arbeitsaufwenden verbunden. \footcite[Vgl.][S. 64]{Wissel2017}  
\section{Ziel der Arbeit}
Im Rahmen dieser Arbeit soll ein Source-To-Source Compiler zwischen den Frameworks Xamarin.Forms und Flutter realisiert werden mit dessen Hilfe die folgende zentrale Forschungsfrage beantwortet werden soll: "Können Apps komplett automatisiert von Xamarin.Forms zu Flutter übersetzt werden, oder sind manuelle Arbeitsschritte erforderlich?"
\section{Motivation}
Im Mai 2020 hat Microsoft mit .NET MAUI einen Nachfolger für das Xamarin.Forms Framework angekündigt, der im Herbst 2021 zusammen mit der sechsten Version des .NET Frameworks veröffentlicht werden soll. Zum aktuellen Zeitpunkt ist bereits angekündigt, dass .NET MAUI grundlegende Änderung mit sich bringt und Anwendungen die mit Hilfe von Xamarin.Forms entwickelt wurden angepasst werden müssen.\footcite[Vgl.][Abgerufen am 28.10.2020]{Hunter2020}

Für Xamarin.Forms Entwickler wird es also unausweichlich sein, grundlegende Änderungen an bereits realisierten Anwendungen vorzunehmen um in der Zukunft von Aktualisierungen zu profitieren. Aus diesem Grund eignet sich die Zeit, um auch alternative Frameworks zu analysieren und zu überprüfen ob ein  Umstieg lohnenswert ist. Das Flutter Framework erfreut sich unter Entwicklern einer wachsenden Beliebtheit und ist nicht nur auf Grund seiner performanten Anwendungen mittlerweile weit verbreitet. 

Ein automatisierter Umstieg auf das von Google entwickelte Framework würde also nicht nur die Anpassungen an .NET Maui verhindern, sondern auch leistungsfähigere Anwendungen generieren. 

\section{Gliederrung}

  \include{02SourceToSourceCompiler}
  \chapter{Cross Plattform Frameworks}
Für die Realisierung eine Source-to-Source Compilers gibt es folglich zwei Faktoren,  die für die Realisierung ausschlaggebend sind.  Zum einen die Programmiersprachen in dem die beiden Frameworks entwickelt werden,  da bei der Übersetzung eine Brücke für die Übersetzung zwischen Quell und Zielsprache verwendet werden muss.  Neben der Programmiersprache ist jedoch auch die Arbeitsweise eines Frameworks von essentieller Bedeutung.  Wie die Definition von Compilern bereits einführte,  müssen die Programme vor und nach der Übersetzung gleichwertig sein.  Dies implementiert,  dass das Verhalten der übersetzten Anwendungen nach der Übersetzung identisch seien muss wie das der Ursprungsanwendung.  Es ist also notwendig,  neben den sprachlichen auch die technischen Unterschiede zwischen den Frameworks zu kennen und diese im Rahmen der compilierung zu optimieren. 
\section{Programmiersprachen}
C\# Dart Vergleich
XAML Dart
\section{Frameworks}
Xamarin Forms und Flutter vergleichen
\subsection{Projekte}
Files.
Xamarin.Forms Anwendungen setzen sich aus mehreren Projekten zusammen.  Dabei gibt es klassischerweise für jede Zielpalttform ein Projekt sowie ein zusätzliches für den Quelltext, der zwischen den Plattformen geteilt wird.  Dabei ist das geteilte Projekt eine Abhängigkeit der plattformspezifischen Projekte.  Im Gegensatz dazu gibt es bei Flutter nur ein Projekt, welches alle notwendigen Inhalte für iOS und Android beinhaltet. 
\subsection{Ansichtsseiten}
Ansichtseiten sind visuelle elemente einer Anwendung die den gesamten Bildschirm belegen. (https://docs.microsoft.com/de-de/xamarin/xamarin-forms/user-interface/controls/pages) Bei iOS werden diese klaissischerweise als Viewcontroller und Acitvity bei Android bezeichnet.   Xamarin Forms bietet dafür ein Vielzahl von alternativen an: Eine Contentpage, eine MasterDetailPage, eine NavigationPage,  eine TabbedPage eine TempaltedPage und eine CarouselPage.
Die Auswahl einer Ansischtsseite hat daher einen direkten Einfluss aus die Navigation des Anwenders durch die Verschiedenen Seiten.
Bei Flutter gibt es alternativ weniger Auswahl, da lediglich ein "MaterialApp" und eine "CupertinoApp" als Widget zur Verfügung stellen und alternativ Materialdesign oder Cupertino design anbieten. 
Um Seiten mit Tab oder einem Slideout zu genierieren bietet Flutter jedoch ebenfalls Widgets an. 
\subsection{Elemente}
\subsection{Layouts}
\subsection{Listen}
\subsection{Gesten}
\subsection{Animtion}

\subsection{Navigation}
Pages.
Andere Apps
\subsection{Async UI}
\subsection{Netzwerk Anfragen}
\subsection{Abhängigkeiten}
\subsection{Lebenzyklus}
\subsection{Schriften}
\subsection{Plugins}
Interacting with hardware, third party services, and the platform
\subsection{Databases and locale storage}
\subsection{Notifications}



  \include{04CompilerEntwurf}  
  \include{05Realisierung}
  \chapter{Qualitätssicherung}
\label{chap:Qualitätssicherung}

Nach der erfolgreichen Implementierung des Übersetzers soll mit Hilfe einer zu testenden mobilen Anwendung überprüft werden,  ob der compiler so arbeitet wie zu erwarten ist.  

\section{Testobjekt}
\subsection{Pages}
\subsection{Ansichten}
\subsection{Plattformspezifische Api's}
\section{Testauswertung}



  \include{07Anwendungsfall}
  \chapter{Fazit und Ausblick}
\label{chap:FazitAusblick}

Flutter immernoch eine Abhängigkeit
  
  \appendix                       % Leitet den Nachspann ein
  \pagenumbering{Roman}%  
  \setcounter{page}{6}
  
  \printbibliography              % Setzt das Literaturverzeichnis
  \printEidesstattlicheErklaerung % Setzt die eidesstattlichen Erklärung
\end{document}
